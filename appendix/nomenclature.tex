\clearpage
\chapter{Nomenclature and Acronyms}
\label{sec:Nomenclature_and_Acronyms}

Several packages could be used for this, including \verb|nomencl|\footnote{https://ctan.org/pkg/nomencl?lang=en}
   and \verb|glossaries|\footnote{https://ctan.org/pkg/glossaries?lang=en}.
However, I prefer to sort in my own order and include sub-headings,
   which I found cumbersome with those tools.
Therefore, this nomenclature section just uses longtable.
Feel free to customize as desired!


% -- Nomenclature
\section{Nomenclature}
\label{sec:Nomenclature}

\begin{longtable*}[l]{l @{\qquad} l}

% -- Latin alphabet (no space here because it's first)
                    \bfseries{Latin}            \\
$b$         &  Impact parameter, \si{\angstrom}                            \\
$T$         &  Temperature, \si{\kelvin}                                   \\
$k$         &  Rate constant, \si{\cm\cubed\per\sec}, OR                   \\
            &  Some other constant, not confusing at all                   \\
$Z$         &  Some duplicate entries to wrap                              \\
$Z$         &  Some duplicate entries to wrap                              \\
$Z$         &  Some duplicate entries to wrap                              \\
$Z$         &  Some duplicate entries to wrap                              \\
$Z$         &  Some duplicate entries to wrap                              \\

% -- Greek alphabet
\addlinespace[5 mm] \bfseries{Greek}            \\
$\ev$       &  Vibrational energy, \si{\eV}, \cref{eqn:ev}                 \\

% -- Subscripts
\addlinespace[5 mm] \bfseries{Subscripts}       \\
d           &  Dissociation                                                \\
e           &  Exchange                                                    \\
t           &  Translational                                               \\
r           &  Rotational                                                  \\
v           &  Vibrational
\end{longtable*}

\section{Acronyms}
\label{sec:Acronyms}

\begin{longtable*}[l]{l @{\qquad} l}
NASA        &  National Aeronautics and Space Administration               \\
PES         &  Potential Energy Surface
\end{longtable*}
